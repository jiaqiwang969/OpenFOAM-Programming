\documentclass{beamer}
\usepackage{ctex, hyperref}
\usepackage[T1]{fontenc}
\usepackage{listings} %c++插入代码
\lstset{language=C++}%这条命令可以让LaTeX排版时将C++键字突出显示
\lstset{breaklines}%这条命令可以让LaTeX自动将长的代码行换行排版
%\lstset{extendedchars=false}%这一条命令可以解决代码跨页时,章节标题,页眉等汉字不显示的问题
% \usepackage{animate}
\usepackage{verbatim} %批量注释
\newcommand{\SubItem}[1]{
    {\setlength\itemindent{10pt} \item[-] #1}
}



% other packages
\usepackage{latexsym,amsmath,xcolor,multicol,booktabs,calligra}
\usepackage{graphicx,pstricks,listings,stackengine}

\author{{王佳琪}}
\title{Introduction to OpenFOAM Programming}
\subtitle{01 - C++数据结构}
\institute{\kaishu{上海交通大学}}
\date{\kaishu{2022年1月}}
\usepackage{WUT}

% defs
\def\cmd#1{\texttt{\color{red}\footnotesize $\backslash$#1}}
\def\env#1{\texttt{\color{blue}\footnotesize #1}}
\definecolor{deepblue}{rgb}{0,0,0.5}
\definecolor{deepred}{rgb}{0.6,0,0}
\definecolor{deepgreen}{rgb}{0,0.5,0}
\definecolor{halfgray}{gray}{0.55}

\lstset{
    basicstyle=\ttfamily\small,
    keywordstyle=\bfseries\color{deepblue},
    emphstyle=\ttfamily\color{deepred},    % Custom highlighting style
    stringstyle=\color{deepgreen},
    numbers=left,
    numberstyle=\small\color{halfgray},
    rulesepcolor=\color{red!20!green!20!blue!20},
    frame=shadowbox,
}


\begin{document}

\songti
\begin{frame}
    \titlepage
    \begin{figure}[htpb]
        \begin{center}
            \includegraphics[width=0.15\linewidth]{pic/WUT.png}
        \end{center}
    \end{figure}
\end{frame}

\begin{frame}
    \tableofcontents[sectionstyle=show,subsectionstyle=show/shaded/hide,subsubsectionstyle=show/shaded/hide]
\end{frame}





%%~~~~~~~~~~~~~~~~~~~~~正文~~~~~~~~~~~~~~~~~~%%

\section{什么是数据结构?}


\include{01-datastruct/00-dataStruct/04-非线性数据结构}





\subsection{02-List}
\include{01-datastruct/02-list/00-list}
\include{01-datastruct/02-list/01-constructor}
% \include{01-datastruct/02-list/02-assign}
% \include{01-datastruct/02-list/03-insert}
% \include{01-datastruct/02-list/04-swap}
% \include{01-datastruct/02-list/05-emplace_back}
% \include{01-datastruct/02-list/05-emplace_front}
% % 条件判断成员函数,在list上线性效率。
% \include{01-datastruct/02-list/06-sort} %NlogN
% \include{01-datastruct/02-list/07-merge}
% \include{01-datastruct/02-list/08-remove}
% \include{01-datastruct/02-list/08-remove_if}
% \include{01-datastruct/02-list/09-splice}
% \include{01-datastruct/02-list/10-reverse}


% \subsection{03-forwardList}
% \include{01-datastruct/03-forwordlist/00-forwardlist}
% \include{01-datastruct/03-forwordlist/01-constructor}





\section{非线性数据结构}
\subsection{03-tree}
%参考:https://blog.csdn.net/qq_21989927/category_10207344.html
% \include{01-datastruct/04-tree/01-binaryTree/..}
% \include{01-datastruct/04-tree/02-BinarySearchTree/..}
% \include{01-datastruct/04-tree/03-HeapTree/..}
% \include{01-datastruct/04-tree/04-Red-blackTree/..}




\end{document}


