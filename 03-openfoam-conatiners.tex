\documentclass{beamer}
\usepackage{ctex, hyperref}
\usepackage[T1]{fontenc}
\usepackage{listings} %c++插入代码
\lstset{language=C++}%这条命令可以让LaTeX排版时将C++键字突出显示
\lstset{breaklines}%这条命令可以让LaTeX自动将长的代码行换行排版
%\lstset{extendedchars=false}%这一条命令可以解决代码跨页时,章节标题,页眉等汉字不显示的问题
% \usepackage{animate}
\usepackage{verbatim} %批量注释
\newcommand{\SubItem}[1]{
    {\setlength\itemindent{10pt} \item[-] #1}
}

\def\pgfsysdriver{pgfsys -dvipdfmx.def}
\usepackage{tikz} %画图工具




% other packages
\usepackage{latexsym,amsmath,xcolor,multicol,booktabs,calligra}
\usepackage{graphicx,pstricks,listings,stackengine}

\author{{王佳琪}}
\title{Introduction to OpenFOAM Programming}
\subtitle{03 - Openfoam 数据结构}
\institute{\kaishu{上海交通大学}}
\date{\kaishu{2022年1月}}
\usepackage{WUT}

% defs
\def\cmd#1{\texttt{\color{red}\footnotesize $\backslash$#1}}
\def\env#1{\texttt{\color{blue}\footnotesize #1}}
\definecolor{deepblue}{rgb}{0,0,0.5}
\definecolor{deepred}{rgb}{0.6,0,0}
\definecolor{deepgreen}{rgb}{0,0.5,0}
\definecolor{halfgray}{gray}{0.55}

\lstset{
    basicstyle=\ttfamily\small,
    keywordstyle=\bfseries\color{deepblue},
    emphstyle=\ttfamily\color{deepred},    % Custom highlighting style
    stringstyle=\color{deepgreen},
    numbers=left,
    numberstyle=\small\color{halfgray},
    rulesepcolor=\color{red!20!green!20!blue!20},
    frame=shadowbox,
}


\begin{document}

\songti
\begin{frame}
    \titlepage
    \begin{figure}[htpb]
        \begin{center}
            \includegraphics[width=0.15\linewidth]{pic/WUT.png}
        \end{center}
    \end{figure}
\end{frame}

\begin{frame}
    \tableofcontents[sectionstyle=show,subsectionstyle=show/shaded/hide,subsubsectionstyle=show/shaded/hide]
\end{frame}





%%~~~~~~~~~~~~~~~~~~~~~正文~~~~~~~~~~~~~~~~~~%%

\section{primitives基础类}

\include{03-primitives/00-总结}
\include{03-primitives/01-int}
\include{03-primitives/01-label}
\include{03-primitives/02-char}
\include{03-primitives/02-string}
\include{03-primitives/03-List}
\include{03-primitives/03-HashTable}
\include{03-primitives/04-scalar}
\include{03-primitives/04-vector}
\include{03-primitives/04-tensor}

\include{03-primitives/05-field}

\section{openfoam class}
% \begin{figure}[htpb]
%     \includegraphics[width=1 \linewidth]{pic/04-架构图.png}
%     % \caption{ \href{https://cpp.openfoam.org/v9/classFoam_1_1regIOobject.html}{\color{purple}{regIOobject 类继承关系}}}
% \end{figure}

- Space and time: polyMesh, fvMesh, Time\\
- Field algebra: Field, DimensionedField and GeometricField\\
- Boundary conditions: fvPatchField and derived classes\\
- Sparse matrices: IduMatrix, fvMatrix and linear solvers\\
- Finite Volume discretisation: fvc and fvm namespace\\
% https://www.youtube.com/watch?v=6NjE06lX9js&list=PLtlltkn_UCCzF7aGmBZ66CzK48ipi1El6&index=3

\section{openfoam模式设计}

\include{03-primitives/06-lookup}
\include{03-primitives/07-ObjectRegistry}









% \include{03-primitives/t1}










% \begin{frame}[fragile]{Circulator}
%     \begin{figure}[htpb]
%         \centering
%         \begin{tikzpicture}

%             \def \n {4}
%             \def \radius {1cm}
%             \def \margin {20} % margin in angles, depends on the radius

%             \foreach \s in {1,...,\n}
%             {
%             \node[draw, circle] at ({360/\n * (\s - 1)}:\radius) {\s};
%             \draw[->, >=latex] ({360/\n * (\s - 1)+\margin}:\radius)
%             arc ({360/\n * (\s - 1)+\margin}:{360/\n * (\s)-\margin}:\radius);
%             }
%         \end{tikzpicture}

%         \begin{tikzpicture}
%             \tikzset{
%                 box/.style ={
%                         rectangle, %矩形节点
%                         rounded corners =5pt, %圆角
%                         minimum width =50pt, %最小宽度
%                         minimum height =20pt, %最小高度
%                         inner sep=5pt, %文字和边框的距离
%                         draw=blue %边框颜色}
%                     }
%             }
%             \node[box] (1) at(0,0) {cStart.prev()};
%             \node[box] (2) at(3,0) {cStart()};
%             \node[box] (3) at(6,0) {cStart.next()};
%             \draw[->] (1)--(2);
%             \draw[->] (2)--(3);
%             % \node at(2,1) {a};
%             % \node at(6,1) {b};
%         \end{tikzpicture}
%         \caption{Circulator数据结构}
%     \end{figure}

%     \begin{lstlisting}[basicstyle=\tiny]
%     cStart.circulate(CirculatorBase::direction::clockwise)
%     cStart.size() == 4
%       \end{lstlisting}

%     %%04-doc:
%     \tiny{
%         \href {

%             https://github.com/OpenFOAM/OpenFOAM-9/blob/master/applications/test/Circulator/Test-Circulator.C

%         }{\color{purple}{02-doc}}}


% \end{frame}


% \begin{frame}[fragile]{CompactIOList for faceList}
%     \begin{minipage}{0.45\linewidth}
%         \begin{lstlisting}[basicstyle=\tiny,caption={Old format},captionpos=b]
% faceIOList faces2
% (
%     IOobject
%     (
%         "faces2",
%         runTime.constant(),
%         polyMesh::meshSubDir,
%         runTime,
%         IOobject::NO_READ,
%         IOobject::NO_WRITE,
%         false
%     ),
%     size
% );
%         \end{lstlisting}
%     \end{minipage}
%     %
%     \begin{minipage}{0.45\linewidth}
%         \begin{lstlisting}[basicstyle=\tiny,caption={New format},captionpos=b]
% faceCompactIOList faces2
% (
%     IOobject
%     (
%         "faces2",
%         runTime.constant(),
%         polyMesh::meshSubDir,
%         runTime,
%         IOobject::NO_READ,
%         IOobject::NO_WRITE,
%         false
%     ),
%     size
% );
% \end{lstlisting}
%     \end{minipage}


%     %%04-doc:
%     \tiny{
%         \href {

%             https://github.com/OpenFOAM/OpenFOAM-9/blob/master/applications/test/CompactIOList/Test-CompactIOList.C

%         }{\color{purple}{02-doc}}}

% \end{frame}



% \begin{frame}[fragile]{CompactListList}





%     %%04-doc:
%     \tiny{
%         \href {

%             https://github.com/OpenFOAM/OpenFOAM-9/blob/master/applications/test/CompactListList/Test-CompactListList.C

%         }{\color{purple}{02-doc}}}


% \end{frame}
% \include{01-datastruct/00-dataStruct/01-memory}
% \include{01-datastruct/00-dataStruct/02-iterator}
% \include{01-datastruct/00-dataStruct/00-dataStruct}
% \include{01-datastruct/00-dataStruct/03-线性数据结构}
% \include{01-datastruct/00-dataStruct/04-非线性数据结构}



\end{document}


